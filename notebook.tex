
% Default to the notebook output style

    


% Inherit from the specified cell style.




    
\documentclass[11pt]{article}

    
    
    \usepackage[T1]{fontenc}
    % Nicer default font (+ math font) than Computer Modern for most use cases
    \usepackage{mathpazo}

    % Basic figure setup, for now with no caption control since it's done
    % automatically by Pandoc (which extracts ![](path) syntax from Markdown).
    \usepackage{graphicx}
    % We will generate all images so they have a width \maxwidth. This means
    % that they will get their normal width if they fit onto the page, but
    % are scaled down if they would overflow the margins.
    \makeatletter
    \def\maxwidth{\ifdim\Gin@nat@width>\linewidth\linewidth
    \else\Gin@nat@width\fi}
    \makeatother
    \let\Oldincludegraphics\includegraphics
    % Set max figure width to be 80% of text width, for now hardcoded.
    \renewcommand{\includegraphics}[1]{\Oldincludegraphics[width=.8\maxwidth]{#1}}
    % Ensure that by default, figures have no caption (until we provide a
    % proper Figure object with a Caption API and a way to capture that
    % in the conversion process - todo).
    \usepackage{caption}
    \DeclareCaptionLabelFormat{nolabel}{}
    \captionsetup{labelformat=nolabel}

    \usepackage{adjustbox} % Used to constrain images to a maximum size 
    \usepackage{xcolor} % Allow colors to be defined
    \usepackage{enumerate} % Needed for markdown enumerations to work
    \usepackage{geometry} % Used to adjust the document margins
    \usepackage{amsmath} % Equations
    \usepackage{amssymb} % Equations
    \usepackage{textcomp} % defines textquotesingle
    % Hack from http://tex.stackexchange.com/a/47451/13684:
    \AtBeginDocument{%
        \def\PYZsq{\textquotesingle}% Upright quotes in Pygmentized code
    }
    \usepackage{upquote} % Upright quotes for verbatim code
    \usepackage{eurosym} % defines \euro
    \usepackage[mathletters]{ucs} % Extended unicode (utf-8) support
    \usepackage[utf8x]{inputenc} % Allow utf-8 characters in the tex document
    \usepackage{fancyvrb} % verbatim replacement that allows latex
    \usepackage{grffile} % extends the file name processing of package graphics 
                         % to support a larger range 
    % The hyperref package gives us a pdf with properly built
    % internal navigation ('pdf bookmarks' for the table of contents,
    % internal cross-reference links, web links for URLs, etc.)
    \usepackage{hyperref}
    \usepackage{longtable} % longtable support required by pandoc >1.10
    \usepackage{booktabs}  % table support for pandoc > 1.12.2
    \usepackage[inline]{enumitem} % IRkernel/repr support (it uses the enumerate* environment)
    \usepackage[normalem]{ulem} % ulem is needed to support strikethroughs (\sout)
                                % normalem makes italics be italics, not underlines
    

    
    
    % Colors for the hyperref package
    \definecolor{urlcolor}{rgb}{0,.145,.698}
    \definecolor{linkcolor}{rgb}{.71,0.21,0.01}
    \definecolor{citecolor}{rgb}{.12,.54,.11}

    % ANSI colors
    \definecolor{ansi-black}{HTML}{3E424D}
    \definecolor{ansi-black-intense}{HTML}{282C36}
    \definecolor{ansi-red}{HTML}{E75C58}
    \definecolor{ansi-red-intense}{HTML}{B22B31}
    \definecolor{ansi-green}{HTML}{00A250}
    \definecolor{ansi-green-intense}{HTML}{007427}
    \definecolor{ansi-yellow}{HTML}{DDB62B}
    \definecolor{ansi-yellow-intense}{HTML}{B27D12}
    \definecolor{ansi-blue}{HTML}{208FFB}
    \definecolor{ansi-blue-intense}{HTML}{0065CA}
    \definecolor{ansi-magenta}{HTML}{D160C4}
    \definecolor{ansi-magenta-intense}{HTML}{A03196}
    \definecolor{ansi-cyan}{HTML}{60C6C8}
    \definecolor{ansi-cyan-intense}{HTML}{258F8F}
    \definecolor{ansi-white}{HTML}{C5C1B4}
    \definecolor{ansi-white-intense}{HTML}{A1A6B2}

    % commands and environments needed by pandoc snippets
    % extracted from the output of `pandoc -s`
    \providecommand{\tightlist}{%
      \setlength{\itemsep}{0pt}\setlength{\parskip}{0pt}}
    \DefineVerbatimEnvironment{Highlighting}{Verbatim}{commandchars=\\\{\}}
    % Add ',fontsize=\small' for more characters per line
    \newenvironment{Shaded}{}{}
    \newcommand{\KeywordTok}[1]{\textcolor[rgb]{0.00,0.44,0.13}{\textbf{{#1}}}}
    \newcommand{\DataTypeTok}[1]{\textcolor[rgb]{0.56,0.13,0.00}{{#1}}}
    \newcommand{\DecValTok}[1]{\textcolor[rgb]{0.25,0.63,0.44}{{#1}}}
    \newcommand{\BaseNTok}[1]{\textcolor[rgb]{0.25,0.63,0.44}{{#1}}}
    \newcommand{\FloatTok}[1]{\textcolor[rgb]{0.25,0.63,0.44}{{#1}}}
    \newcommand{\CharTok}[1]{\textcolor[rgb]{0.25,0.44,0.63}{{#1}}}
    \newcommand{\StringTok}[1]{\textcolor[rgb]{0.25,0.44,0.63}{{#1}}}
    \newcommand{\CommentTok}[1]{\textcolor[rgb]{0.38,0.63,0.69}{\textit{{#1}}}}
    \newcommand{\OtherTok}[1]{\textcolor[rgb]{0.00,0.44,0.13}{{#1}}}
    \newcommand{\AlertTok}[1]{\textcolor[rgb]{1.00,0.00,0.00}{\textbf{{#1}}}}
    \newcommand{\FunctionTok}[1]{\textcolor[rgb]{0.02,0.16,0.49}{{#1}}}
    \newcommand{\RegionMarkerTok}[1]{{#1}}
    \newcommand{\ErrorTok}[1]{\textcolor[rgb]{1.00,0.00,0.00}{\textbf{{#1}}}}
    \newcommand{\NormalTok}[1]{{#1}}
    
    % Additional commands for more recent versions of Pandoc
    \newcommand{\ConstantTok}[1]{\textcolor[rgb]{0.53,0.00,0.00}{{#1}}}
    \newcommand{\SpecialCharTok}[1]{\textcolor[rgb]{0.25,0.44,0.63}{{#1}}}
    \newcommand{\VerbatimStringTok}[1]{\textcolor[rgb]{0.25,0.44,0.63}{{#1}}}
    \newcommand{\SpecialStringTok}[1]{\textcolor[rgb]{0.73,0.40,0.53}{{#1}}}
    \newcommand{\ImportTok}[1]{{#1}}
    \newcommand{\DocumentationTok}[1]{\textcolor[rgb]{0.73,0.13,0.13}{\textit{{#1}}}}
    \newcommand{\AnnotationTok}[1]{\textcolor[rgb]{0.38,0.63,0.69}{\textbf{\textit{{#1}}}}}
    \newcommand{\CommentVarTok}[1]{\textcolor[rgb]{0.38,0.63,0.69}{\textbf{\textit{{#1}}}}}
    \newcommand{\VariableTok}[1]{\textcolor[rgb]{0.10,0.09,0.49}{{#1}}}
    \newcommand{\ControlFlowTok}[1]{\textcolor[rgb]{0.00,0.44,0.13}{\textbf{{#1}}}}
    \newcommand{\OperatorTok}[1]{\textcolor[rgb]{0.40,0.40,0.40}{{#1}}}
    \newcommand{\BuiltInTok}[1]{{#1}}
    \newcommand{\ExtensionTok}[1]{{#1}}
    \newcommand{\PreprocessorTok}[1]{\textcolor[rgb]{0.74,0.48,0.00}{{#1}}}
    \newcommand{\AttributeTok}[1]{\textcolor[rgb]{0.49,0.56,0.16}{{#1}}}
    \newcommand{\InformationTok}[1]{\textcolor[rgb]{0.38,0.63,0.69}{\textbf{\textit{{#1}}}}}
    \newcommand{\WarningTok}[1]{\textcolor[rgb]{0.38,0.63,0.69}{\textbf{\textit{{#1}}}}}
    
    
    % Define a nice break command that doesn't care if a line doesn't already
    % exist.
    \def\br{\hspace*{\fill} \\* }
    % Math Jax compatability definitions
    \def\gt{>}
    \def\lt{<}
    % Document parameters
    \title{pyNrNotebook1}
    
    
    

    % Pygments definitions
    
\makeatletter
\def\PY@reset{\let\PY@it=\relax \let\PY@bf=\relax%
    \let\PY@ul=\relax \let\PY@tc=\relax%
    \let\PY@bc=\relax \let\PY@ff=\relax}
\def\PY@tok#1{\csname PY@tok@#1\endcsname}
\def\PY@toks#1+{\ifx\relax#1\empty\else%
    \PY@tok{#1}\expandafter\PY@toks\fi}
\def\PY@do#1{\PY@bc{\PY@tc{\PY@ul{%
    \PY@it{\PY@bf{\PY@ff{#1}}}}}}}
\def\PY#1#2{\PY@reset\PY@toks#1+\relax+\PY@do{#2}}

\expandafter\def\csname PY@tok@w\endcsname{\def\PY@tc##1{\textcolor[rgb]{0.73,0.73,0.73}{##1}}}
\expandafter\def\csname PY@tok@c\endcsname{\let\PY@it=\textit\def\PY@tc##1{\textcolor[rgb]{0.25,0.50,0.50}{##1}}}
\expandafter\def\csname PY@tok@cp\endcsname{\def\PY@tc##1{\textcolor[rgb]{0.74,0.48,0.00}{##1}}}
\expandafter\def\csname PY@tok@k\endcsname{\let\PY@bf=\textbf\def\PY@tc##1{\textcolor[rgb]{0.00,0.50,0.00}{##1}}}
\expandafter\def\csname PY@tok@kp\endcsname{\def\PY@tc##1{\textcolor[rgb]{0.00,0.50,0.00}{##1}}}
\expandafter\def\csname PY@tok@kt\endcsname{\def\PY@tc##1{\textcolor[rgb]{0.69,0.00,0.25}{##1}}}
\expandafter\def\csname PY@tok@o\endcsname{\def\PY@tc##1{\textcolor[rgb]{0.40,0.40,0.40}{##1}}}
\expandafter\def\csname PY@tok@ow\endcsname{\let\PY@bf=\textbf\def\PY@tc##1{\textcolor[rgb]{0.67,0.13,1.00}{##1}}}
\expandafter\def\csname PY@tok@nb\endcsname{\def\PY@tc##1{\textcolor[rgb]{0.00,0.50,0.00}{##1}}}
\expandafter\def\csname PY@tok@nf\endcsname{\def\PY@tc##1{\textcolor[rgb]{0.00,0.00,1.00}{##1}}}
\expandafter\def\csname PY@tok@nc\endcsname{\let\PY@bf=\textbf\def\PY@tc##1{\textcolor[rgb]{0.00,0.00,1.00}{##1}}}
\expandafter\def\csname PY@tok@nn\endcsname{\let\PY@bf=\textbf\def\PY@tc##1{\textcolor[rgb]{0.00,0.00,1.00}{##1}}}
\expandafter\def\csname PY@tok@ne\endcsname{\let\PY@bf=\textbf\def\PY@tc##1{\textcolor[rgb]{0.82,0.25,0.23}{##1}}}
\expandafter\def\csname PY@tok@nv\endcsname{\def\PY@tc##1{\textcolor[rgb]{0.10,0.09,0.49}{##1}}}
\expandafter\def\csname PY@tok@no\endcsname{\def\PY@tc##1{\textcolor[rgb]{0.53,0.00,0.00}{##1}}}
\expandafter\def\csname PY@tok@nl\endcsname{\def\PY@tc##1{\textcolor[rgb]{0.63,0.63,0.00}{##1}}}
\expandafter\def\csname PY@tok@ni\endcsname{\let\PY@bf=\textbf\def\PY@tc##1{\textcolor[rgb]{0.60,0.60,0.60}{##1}}}
\expandafter\def\csname PY@tok@na\endcsname{\def\PY@tc##1{\textcolor[rgb]{0.49,0.56,0.16}{##1}}}
\expandafter\def\csname PY@tok@nt\endcsname{\let\PY@bf=\textbf\def\PY@tc##1{\textcolor[rgb]{0.00,0.50,0.00}{##1}}}
\expandafter\def\csname PY@tok@nd\endcsname{\def\PY@tc##1{\textcolor[rgb]{0.67,0.13,1.00}{##1}}}
\expandafter\def\csname PY@tok@s\endcsname{\def\PY@tc##1{\textcolor[rgb]{0.73,0.13,0.13}{##1}}}
\expandafter\def\csname PY@tok@sd\endcsname{\let\PY@it=\textit\def\PY@tc##1{\textcolor[rgb]{0.73,0.13,0.13}{##1}}}
\expandafter\def\csname PY@tok@si\endcsname{\let\PY@bf=\textbf\def\PY@tc##1{\textcolor[rgb]{0.73,0.40,0.53}{##1}}}
\expandafter\def\csname PY@tok@se\endcsname{\let\PY@bf=\textbf\def\PY@tc##1{\textcolor[rgb]{0.73,0.40,0.13}{##1}}}
\expandafter\def\csname PY@tok@sr\endcsname{\def\PY@tc##1{\textcolor[rgb]{0.73,0.40,0.53}{##1}}}
\expandafter\def\csname PY@tok@ss\endcsname{\def\PY@tc##1{\textcolor[rgb]{0.10,0.09,0.49}{##1}}}
\expandafter\def\csname PY@tok@sx\endcsname{\def\PY@tc##1{\textcolor[rgb]{0.00,0.50,0.00}{##1}}}
\expandafter\def\csname PY@tok@m\endcsname{\def\PY@tc##1{\textcolor[rgb]{0.40,0.40,0.40}{##1}}}
\expandafter\def\csname PY@tok@gh\endcsname{\let\PY@bf=\textbf\def\PY@tc##1{\textcolor[rgb]{0.00,0.00,0.50}{##1}}}
\expandafter\def\csname PY@tok@gu\endcsname{\let\PY@bf=\textbf\def\PY@tc##1{\textcolor[rgb]{0.50,0.00,0.50}{##1}}}
\expandafter\def\csname PY@tok@gd\endcsname{\def\PY@tc##1{\textcolor[rgb]{0.63,0.00,0.00}{##1}}}
\expandafter\def\csname PY@tok@gi\endcsname{\def\PY@tc##1{\textcolor[rgb]{0.00,0.63,0.00}{##1}}}
\expandafter\def\csname PY@tok@gr\endcsname{\def\PY@tc##1{\textcolor[rgb]{1.00,0.00,0.00}{##1}}}
\expandafter\def\csname PY@tok@ge\endcsname{\let\PY@it=\textit}
\expandafter\def\csname PY@tok@gs\endcsname{\let\PY@bf=\textbf}
\expandafter\def\csname PY@tok@gp\endcsname{\let\PY@bf=\textbf\def\PY@tc##1{\textcolor[rgb]{0.00,0.00,0.50}{##1}}}
\expandafter\def\csname PY@tok@go\endcsname{\def\PY@tc##1{\textcolor[rgb]{0.53,0.53,0.53}{##1}}}
\expandafter\def\csname PY@tok@gt\endcsname{\def\PY@tc##1{\textcolor[rgb]{0.00,0.27,0.87}{##1}}}
\expandafter\def\csname PY@tok@err\endcsname{\def\PY@bc##1{\setlength{\fboxsep}{0pt}\fcolorbox[rgb]{1.00,0.00,0.00}{1,1,1}{\strut ##1}}}
\expandafter\def\csname PY@tok@kc\endcsname{\let\PY@bf=\textbf\def\PY@tc##1{\textcolor[rgb]{0.00,0.50,0.00}{##1}}}
\expandafter\def\csname PY@tok@kd\endcsname{\let\PY@bf=\textbf\def\PY@tc##1{\textcolor[rgb]{0.00,0.50,0.00}{##1}}}
\expandafter\def\csname PY@tok@kn\endcsname{\let\PY@bf=\textbf\def\PY@tc##1{\textcolor[rgb]{0.00,0.50,0.00}{##1}}}
\expandafter\def\csname PY@tok@kr\endcsname{\let\PY@bf=\textbf\def\PY@tc##1{\textcolor[rgb]{0.00,0.50,0.00}{##1}}}
\expandafter\def\csname PY@tok@bp\endcsname{\def\PY@tc##1{\textcolor[rgb]{0.00,0.50,0.00}{##1}}}
\expandafter\def\csname PY@tok@fm\endcsname{\def\PY@tc##1{\textcolor[rgb]{0.00,0.00,1.00}{##1}}}
\expandafter\def\csname PY@tok@vc\endcsname{\def\PY@tc##1{\textcolor[rgb]{0.10,0.09,0.49}{##1}}}
\expandafter\def\csname PY@tok@vg\endcsname{\def\PY@tc##1{\textcolor[rgb]{0.10,0.09,0.49}{##1}}}
\expandafter\def\csname PY@tok@vi\endcsname{\def\PY@tc##1{\textcolor[rgb]{0.10,0.09,0.49}{##1}}}
\expandafter\def\csname PY@tok@vm\endcsname{\def\PY@tc##1{\textcolor[rgb]{0.10,0.09,0.49}{##1}}}
\expandafter\def\csname PY@tok@sa\endcsname{\def\PY@tc##1{\textcolor[rgb]{0.73,0.13,0.13}{##1}}}
\expandafter\def\csname PY@tok@sb\endcsname{\def\PY@tc##1{\textcolor[rgb]{0.73,0.13,0.13}{##1}}}
\expandafter\def\csname PY@tok@sc\endcsname{\def\PY@tc##1{\textcolor[rgb]{0.73,0.13,0.13}{##1}}}
\expandafter\def\csname PY@tok@dl\endcsname{\def\PY@tc##1{\textcolor[rgb]{0.73,0.13,0.13}{##1}}}
\expandafter\def\csname PY@tok@s2\endcsname{\def\PY@tc##1{\textcolor[rgb]{0.73,0.13,0.13}{##1}}}
\expandafter\def\csname PY@tok@sh\endcsname{\def\PY@tc##1{\textcolor[rgb]{0.73,0.13,0.13}{##1}}}
\expandafter\def\csname PY@tok@s1\endcsname{\def\PY@tc##1{\textcolor[rgb]{0.73,0.13,0.13}{##1}}}
\expandafter\def\csname PY@tok@mb\endcsname{\def\PY@tc##1{\textcolor[rgb]{0.40,0.40,0.40}{##1}}}
\expandafter\def\csname PY@tok@mf\endcsname{\def\PY@tc##1{\textcolor[rgb]{0.40,0.40,0.40}{##1}}}
\expandafter\def\csname PY@tok@mh\endcsname{\def\PY@tc##1{\textcolor[rgb]{0.40,0.40,0.40}{##1}}}
\expandafter\def\csname PY@tok@mi\endcsname{\def\PY@tc##1{\textcolor[rgb]{0.40,0.40,0.40}{##1}}}
\expandafter\def\csname PY@tok@il\endcsname{\def\PY@tc##1{\textcolor[rgb]{0.40,0.40,0.40}{##1}}}
\expandafter\def\csname PY@tok@mo\endcsname{\def\PY@tc##1{\textcolor[rgb]{0.40,0.40,0.40}{##1}}}
\expandafter\def\csname PY@tok@ch\endcsname{\let\PY@it=\textit\def\PY@tc##1{\textcolor[rgb]{0.25,0.50,0.50}{##1}}}
\expandafter\def\csname PY@tok@cm\endcsname{\let\PY@it=\textit\def\PY@tc##1{\textcolor[rgb]{0.25,0.50,0.50}{##1}}}
\expandafter\def\csname PY@tok@cpf\endcsname{\let\PY@it=\textit\def\PY@tc##1{\textcolor[rgb]{0.25,0.50,0.50}{##1}}}
\expandafter\def\csname PY@tok@c1\endcsname{\let\PY@it=\textit\def\PY@tc##1{\textcolor[rgb]{0.25,0.50,0.50}{##1}}}
\expandafter\def\csname PY@tok@cs\endcsname{\let\PY@it=\textit\def\PY@tc##1{\textcolor[rgb]{0.25,0.50,0.50}{##1}}}

\def\PYZbs{\char`\\}
\def\PYZus{\char`\_}
\def\PYZob{\char`\{}
\def\PYZcb{\char`\}}
\def\PYZca{\char`\^}
\def\PYZam{\char`\&}
\def\PYZlt{\char`\<}
\def\PYZgt{\char`\>}
\def\PYZsh{\char`\#}
\def\PYZpc{\char`\%}
\def\PYZdl{\char`\$}
\def\PYZhy{\char`\-}
\def\PYZsq{\char`\'}
\def\PYZdq{\char`\"}
\def\PYZti{\char`\~}
% for compatibility with earlier versions
\def\PYZat{@}
\def\PYZlb{[}
\def\PYZrb{]}
\makeatother


    % Exact colors from NB
    \definecolor{incolor}{rgb}{0.0, 0.0, 0.5}
    \definecolor{outcolor}{rgb}{0.545, 0.0, 0.0}



    
    % Prevent overflowing lines due to hard-to-break entities
    \sloppy 
    % Setup hyperref package
    \hypersetup{
      breaklinks=true,  % so long urls are correctly broken across lines
      colorlinks=true,
      urlcolor=urlcolor,
      linkcolor=linkcolor,
      citecolor=citecolor,
      }
    % Slightly bigger margins than the latex defaults
    
    \geometry{verbose,tmargin=1in,bmargin=1in,lmargin=1in,rmargin=1in}
    
    

    \begin{document}
    
    
    \maketitle
    
    

    
    \section{R and Python Analytics via Jupyter, Reticulate, and
BigFix}\label{r-and-python-analytics-via-jupyter-reticulate-and-bigfix}

    \subsection{SYNOPSIS:}\label{synopsis}

This notebook (an R session which calls Python 'helper-routines' from
\textbf{configNqueryCmdline.py}) demonstrates interaction between R and
Python programs and dataframes, showing the best features of both
languages.

The program opens and reads credentials \& URL from a configuration-file
(credentials.json), creates a BigFix 'query-channel', and extracts
endpoint information from the BigFix management server, to be
manipulated and graphed using R libraries.

    \begin{Verbatim}[commandchars=\\\{\}]
{\color{incolor}In [{\color{incolor}160}]:} \PY{c+c1}{\PYZsh{} First Look for credentials.json in the parent directory listing}
          \PY{k+kp}{list.files}\PY{p}{(}\PY{l+s}{\PYZsq{}}\PY{l+s}{..\PYZsq{}}\PY{p}{)}
\end{Verbatim}


    \begin{enumerate*}
\item 'add.py'
\item 'baseline\_template.bes'
\item 'baselineActionReport.bes'
\item 'bfClientSetting'
\item 'completedBaseline.bes'
\item 'configNquery'
\item 'configNquery.ipynb'
\item 'configNquery.py'
\item 'createComputerGroup.py'
\item 'createComputerGroup2.py'
\item 'credentials.json'
\item 'cycleGather.py'
\item 'cycleGather.xml'
\item 'doc'
\item 'executeActionList4IEM.py'
\item 'file.xml'
\item 'fixedFixlets.txt'
\item 'getAssetsNCheckinTime.py'
\item 'getFixlets2DictUrllib3.py'
\item 'getFixletsRest.py'
\item 'getFixletsRest2File.py'
\item 'getFixletsRest2File2.py'
\item 'getFixletsRestAdonix.py'
\item 'getFixletsRestAdonixUrllib.py'
\item 'getFixletsRestGrass.py'
\item 'Grasskeet-CriticalVulns-baseline.bes'
\item 'importTweedleDeeFixlet.py'
\item 'importTweedleDeeFixlet3.py'
\item 'invokeBesProvisioning.xml'
\item 'InvokeFixletRestViaPython.docx'
\item 'InvokeFixletRestViaPython.pdf'
\item 'InvokeFixletRestViaPython2.pdf'
\item 'Li3000.txt'
\item 'MasterSite\_FixletA.bes'
\item 'MasterSite\_FixletB.bes'
\item 'MS11-092\_ Vulnerability in Windows Media Could Allow Remote Code Execution - Windows 7 Gold\_SP1 (x64).bes'
\item 'multiActionRestRequest.txt'
\item 'namesXML.txt'
\item 'nHTTP.rb'
\item 'origTemRest.py'
\item 'OtherImportedLibraryStuff'
\item 'postExample.txt'
\item 'protoComputerGroup.xml'
\item 'raw\_group.txt'
\item 'readConfigFmR'
\item 'readFile2String.py'
\item 'readFixletsIntoDictFormatNdisplay.py'
\item 'README.md'
\item 'restSnippet.py'
\item 'rIEMifViaRest.py'
\item 'sampleComputerGroup.bes'
\item 'sampleComputerGroup.xml'
\item 'SCAexamples'
\item 'ServerAutomationExamples'
\item 'SpybotTasks.bes'
\item 'SpybotTasks.xml'
\item 'startBesGather.xml'
\item 'Stop \& Start BESGather.bes'
\item 'Stop BesGather service.bes'
\item 'stopBesGather.txt'
\item 'stopBesGather.xml'
\item 'stopGather.py'
\item 'SUA-InventoryExamples'
\item 'target\_computer\_group.txt'
\item 'temPostActionViaRest.py'
\item 'temPostInvokeProvisioning.py'
\item 'temPostInvokeProvisioning2.py'
\item 'testPostGroup.xml'
\item 'testPostParameter.py'
\item 'testPostParameter.xml'
\item 'tokenFile.txt'
\item 'TweedleDee.bes'
\item 'Untitled.ipynb'
\end{enumerate*}


    
    \textbf{install/instantiate Reticulate library using
install.packages('reticulate') and library('reticulate'), then 'pull in'
configNqueryCmdline.py}

    \begin{Verbatim}[commandchars=\\\{\}]
{\color{incolor}In [{\color{incolor}171}]:} \PY{c+c1}{\PYZsh{} install.packages(\PYZsq{}reticulate\PYZsq{})}
          \PY{k+kn}{library}\PY{p}{(}\PY{l+s}{\PYZdq{}}\PY{l+s}{reticulate\PYZdq{}}\PY{p}{)}
          source\PYZus{}python\PY{p}{(}\PY{l+s}{\PYZdq{}}\PY{l+s}{rdConfigNqueryBF.py\PYZdq{}}\PY{p}{)}
\end{Verbatim}


    \textbf{As a first pass, just get some computer names from BigFix and
print them out}

    \begin{Verbatim}[commandchars=\\\{\}]
{\color{incolor}In [{\color{incolor}172}]:} \PY{c+c1}{\PYZsh{} First\PYZhy{}Pass: Set up and read the credentials\PYZhy{}file, returning the configuration\PYZhy{}dictionary}
          rCfg\PY{o}{\PYZlt{}\PYZhy{}}\PY{l+s}{\PYZdq{}}\PY{l+s}{../credentials.json\PYZdq{}}
          myCfg\PY{o}{\PYZlt{}\PYZhy{}}readConfig\PY{p}{(}rCfg\PY{p}{)}
          
          \PY{c+c1}{\PYZsh{} Create a \PYZsq{}relevance string and query the BigFix server about computers containing \PYZsq{}adhay\PYZsq{} in their hostnames}
          newRelevance\PY{o}{\PYZlt{}\PYZhy{}}\PY{l+s}{\PYZdq{}}\PY{l+s}{names of bes computers\PYZdq{}}
          
          \PY{c+c1}{\PYZsh{} Use configuration and relevance to query BigFix}
          xmlResult\PY{o}{\PYZlt{}\PYZhy{}}queryBFviaRelevance\PY{p}{(}myCfg\PY{p}{,}newRelevance\PY{p}{)}
          \PY{c+c1}{\PYZsh{}print(xmlResult)}
          
          \PY{c+c1}{\PYZsh{} Using a lambda function defined for the purpose of parsing the bes computers XML, }
          \PY{c+c1}{\PYZsh{} print out the first 6 elements of the DataFrame}
          \PY{k+kp}{print}\PY{p}{(}\PY{k+kp}{head}\PY{p}{(}computersLf1\PY{p}{(}xmlResult\PY{p}{)}\PY{p}{)}\PY{p}{)}
\end{Verbatim}


    \begin{Verbatim}[commandchars=\\\{\}]
           0
1    A206915
2    A206904
3  VBT222-28
4    A122443
5 ADHAYGAP21
6    A206607

    \end{Verbatim}

    \textbf{As a second pass, get a particular set of computer names, but
also include their last 'checkin time'}

    \begin{Verbatim}[commandchars=\\\{\}]
{\color{incolor}In [{\color{incolor}173}]:} \PY{c+c1}{\PYZsh{} Second\PYZhy{}Pass:}
          \PY{c+c1}{\PYZsh{} Create a \PYZsq{}relevance string and query the BigFix server }
          \PY{c+c1}{\PYZsh{} We\PYZsq{}re looking for all endpoints containing \PYZdq{}ADHAY\PYZdq{} in the name and their last checkin time}
          \PY{c+c1}{\PYZsh{} (notice encoding of double\PYZhy{}quote and ampersand characters)}
          newRelevance\PY{o}{\PYZlt{}\PYZhy{}}\PY{l+s}{\PYZdq{}}\PY{l+s}{(name of it \PYZpc{}26 \PYZpc{}22\PYZgt{}\PYZpc{}22 \PYZpc{}26 (last report time of it) as string) of bes computers whose (name of it as lowercase contains \PYZpc{}22adhay\PYZpc{}22)\PYZdq{}}
          xmlResult\PY{o}{\PYZlt{}\PYZhy{}}queryBFviaRelevance\PY{p}{(}myCfg\PY{p}{,}newRelevance\PY{p}{)}
          
          \PY{c+c1}{\PYZsh{} Using the second lambda function we defined for parsing bes computers, }
          \PY{c+c1}{\PYZsh{} return a dataframe of tuples split by \PYZdq{}\PYZgt{}\PYZdq{}}
          df0\PY{o}{\PYZlt{}\PYZhy{}}computersLf2\PY{p}{(}xmlResult\PY{p}{)}
          
          \PY{c+c1}{\PYZsh{} Now clean up the data \PYZam{} rename the columns appropriately, displaying a sample}
          \PY{k+kp}{names}\PY{p}{(}df0\PY{p}{)}\PY{o}{\PYZlt{}\PYZhy{}}\PY{k+kt}{c}\PY{p}{(}\PY{l+s}{\PYZdq{}}\PY{l+s}{Endpoint Name\PYZdq{}}\PY{p}{,}\PY{l+s}{\PYZdq{}}\PY{l+s}{Last Checkin Time\PYZdq{}}\PY{p}{)}
          \PY{k+kp}{head}\PY{p}{(}df0\PY{p}{)}
          
          \PY{c+c1}{\PYZsh{} Let\PYZsq{}s see how many tuples we have...}
          str\PY{p}{(}df0\PY{p}{)}
\end{Verbatim}


    \begin{tabular}{r|ll}
 Endpoint Name & Last Checkin Time\\
\hline
	 ADHAYGAP21                      & Wed, 18 Jul 2018 11:36:29 -0700\\
	 ADHAYVCR01                      & Wed, 18 Jul 2018 11:39:36 -0700\\
	 ADHAYFDB01                      & Wed, 18 Jul 2018 11:03:06 -0700\\
	 ADHAYMCS01N1                    & Wed, 18 Jul 2018 11:31:05 -0700\\
	 ADHAYGAP16                      & Wed, 18 Jul 2018 11:40:01 -0700\\
	 ADHAYWEB08                      & Wed, 18 Jul 2018 11:30:22 -0700\\
\end{tabular}


    
    \begin{Verbatim}[commandchars=\\\{\}]
'data.frame':	144 obs. of  2 variables:
 \$ Endpoint Name    : chr  "ADHAYGAP21" "ADHAYVCR01" "ADHAYFDB01" "ADHAYMCS01N1" {\ldots}
 \$ Last Checkin Time: chr  "Wed, 18 Jul 2018 11:36:29 -0700" "Wed, 18 Jul 2018 11:39:36 -0700" "Wed, 18 Jul 2018 11:03:06 -0700" "Wed, 18 Jul 2018 11:31:05 -0700" {\ldots}
 - attr(*, "pandas.index")=RangeIndex(start=0, stop=144, step=1)

    \end{Verbatim}

    \textbf{Finally, let's get both their 'Last Report Time' and their
'First Report Time', displayed in a table}

    \begin{Verbatim}[commandchars=\\\{\}]
{\color{incolor}In [{\color{incolor}174}]:} \PY{c+c1}{\PYZsh{} Third\PYZhy{}Pass:}
          \PY{c+c1}{\PYZsh{} Create a \PYZsq{}relevance string and query the BigFix server for all endpoints containing \PYZdq{}ADHAY\PYZdq{} in the name, }
          \PY{c+c1}{\PYZsh{} their last checkin time, and initial install date}
          
          \PY{c+c1}{\PYZsh{} BTW, (notice the encoding of double\PYZhy{}quotes (\PYZpc{}22), \PYZdq{}\PYZgt{}\PYZdq{} (\PYZpc{}3e), and ampersand (\PYZpc{}26) characters)}
          newRelevance\PY{o}{\PYZlt{}\PYZhy{}}\PY{l+s}{\PYZdq{}}\PY{l+s}{(name of it \PYZpc{}26 \PYZpc{}22\PYZpc{}3e\PYZpc{}22 \PYZpc{}26 last report time of it as string \PYZpc{}26 \PYZpc{}22\PYZpc{}3e\PYZpc{}22 \PYZpc{}26 (value of results (it, bes property \PYZpc{}22InitialInstallDate\PYZpc{}22) as string)) of bes computers whose}
          \PY{l+s}{(name of it as lowercase contains \PYZpc{}22adhay\PYZpc{}22 and exists value of it of results (it, bes property \PYZpc{}22InitialInstallDate\PYZpc{}22))\PYZdq{}}
          
          xmlResult\PY{o}{\PYZlt{}\PYZhy{}}queryBFviaRelevance\PY{p}{(}myCfg\PY{p}{,}newRelevance\PY{p}{)}
          
          \PY{c+c1}{\PYZsh{} Using the second lambda function we defined for parsing bes computers, }
          \PY{c+c1}{\PYZsh{} clean up the dataframe renaming columns to something descriptive, \PYZam{} display the first 6}
          df0\PY{o}{\PYZlt{}\PYZhy{}}computersLf2\PY{p}{(}xmlResult\PY{p}{)}
          \PY{k+kp}{names}\PY{p}{(}df0\PY{p}{)}\PY{o}{\PYZlt{}\PYZhy{}}\PY{k+kt}{c}\PY{p}{(}\PY{l+s}{\PYZdq{}}\PY{l+s}{Endpoint Name\PYZdq{}}\PY{p}{,}\PY{l+s}{\PYZdq{}}\PY{l+s}{Last Checkin Time\PYZdq{}}\PY{p}{,} \PY{l+s}{\PYZdq{}}\PY{l+s}{Initial Install Date\PYZdq{}}\PY{p}{)}
          \PY{k+kp}{head}\PY{p}{(}df0\PY{p}{)}
\end{Verbatim}


    \begin{tabular}{r|lll}
 Endpoint Name & Last Checkin Time & Initial Install Date\\
\hline
	 ADHAYGAP21                      & Wed, 18 Jul 2018 11:36:29 -0700 & Wed, 27 Jul 2016 17:08:32 -0700\\
	 ADHAYVCR01                      & Wed, 18 Jul 2018 11:39:36 -0700 & Wed, 08 Jun 2016 10:33:56 -0700\\
	 ADHAYFDB01                      & Wed, 18 Jul 2018 11:03:06 -0700 & Sun, 20 Sep 2015 08:53:24 -0700\\
	 ADHAYMCS01N1                    & Wed, 18 Jul 2018 11:31:05 -0700 & Sun, 20 Sep 2015 08:39:59 -0700\\
	 ADHAYGAP16                      & Wed, 18 Jul 2018 11:40:01 -0700 & Thu, 16 Mar 2017 09:38:15 -0700\\
	 ADHAYWEB08                      & Wed, 18 Jul 2018 11:30:22 -0700 & Tue, 02 May 2017 07:43:19 -0700\\
\end{tabular}


    
    \begin{Verbatim}[commandchars=\\\{\}]
{\color{incolor}In [{\color{incolor}188}]:} dateFmt\PY{o}{\PYZlt{}\PYZhy{}}\PY{l+s}{\PYZdq{}}\PY{l+s}{\PYZpc{}a, \PYZpc{}d \PYZpc{}b \PYZpc{}Y \PYZpc{}H:\PYZpc{}M:\PYZpc{}S \PYZpc{}z\PYZdq{}}
\end{Verbatim}


    \begin{Verbatim}[commandchars=\\\{\}]
{\color{incolor}In [{\color{incolor}189}]:} LastReportTime \PY{o}{\PYZlt{}\PYZhy{}}\PY{k+kp}{as.Date}\PY{p}{(}df0\PY{p}{[}\PY{l+m}{1}\PY{p}{,}\PY{l+s}{\PYZdq{}}\PY{l+s}{Last Checkin Time\PYZdq{}}\PY{p}{]}\PY{p}{,}dateFmt\PY{p}{)}
\end{Verbatim}


    \begin{Verbatim}[commandchars=\\\{\}]
{\color{incolor}In [{\color{incolor}190}]:} InitialInstallDate \PY{o}{\PYZlt{}\PYZhy{}} \PY{k+kp}{as.Date}\PY{p}{(}df0\PY{p}{[}\PY{l+m}{1}\PY{p}{,}\PY{l+s}{\PYZdq{}}\PY{l+s}{Initial Install Date\PYZdq{}}\PY{p}{]}\PY{p}{,}dateFmt\PY{p}{)}
\end{Verbatim}


    \begin{Verbatim}[commandchars=\\\{\}]
{\color{incolor}In [{\color{incolor}185}]:} EndpointLifetime \PY{o}{\PYZlt{}\PYZhy{}} LastReportTime \PY{o}{\PYZhy{}} InitialInstallDate
\end{Verbatim}


    \begin{Verbatim}[commandchars=\\\{\}]
{\color{incolor}In [{\color{incolor}187}]:} \PY{k+kp}{head}\PY{p}{(}df0\PY{p}{[}\PY{p}{,}\PY{l+s}{\PYZdq{}}\PY{l+s}{Last Checkin Time\PYZdq{}}\PY{p}{]}\PY{p}{)}
\end{Verbatim}


    \begin{enumerate*}
\item 'Wed, 18 Jul 2018 11:36:29 -0700'
\item 'Wed, 18 Jul 2018 11:39:36 -0700'
\item 'Wed, 18 Jul 2018 11:03:06 -0700'
\item 'Wed, 18 Jul 2018 11:31:05 -0700'
\item 'Wed, 18 Jul 2018 11:40:01 -0700'
\item 'Wed, 18 Jul 2018 11:30:22 -0700'
\end{enumerate*}


    
    \begin{Verbatim}[commandchars=\\\{\}]
{\color{incolor}In [{\color{incolor}206}]:} df0\PY{o}{\PYZdl{}}LastReportTimeDate\PY{o}{\PYZlt{}\PYZhy{}}\PY{k+kp}{as.Date}\PY{p}{(}df0\PY{p}{[}\PY{p}{,}\PY{l+s}{\PYZdq{}}\PY{l+s}{Last Checkin Time\PYZdq{}}\PY{p}{]}\PY{p}{,}dateFmt\PY{p}{)}
          df0\PY{o}{\PYZdl{}}InitialInstallDate\PY{o}{\PYZlt{}\PYZhy{}}\PY{k+kp}{as.Date}\PY{p}{(}df0\PY{p}{[}\PY{p}{,}\PY{l+s}{\PYZdq{}}\PY{l+s}{Initial Install Date\PYZdq{}}\PY{p}{]}\PY{p}{,}dateFmt\PY{p}{)}
          df0\PY{o}{\PYZdl{}}Lifetime\PY{o}{\PYZlt{}\PYZhy{}}df0\PY{o}{\PYZdl{}}LastReportTimeDate\PY{o}{\PYZhy{}}df0\PY{o}{\PYZdl{}}InitialInstallDate
          df0\PY{o}{\PYZdl{}}LastReportTimeDate\PY{o}{\PYZlt{}\PYZhy{}}\PY{k+kc}{NULL}
          df0\PY{o}{\PYZdl{}}InitialInstallDate\PY{o}{\PYZlt{}\PYZhy{}}\PY{k+kc}{NULL}
          \PY{k+kp}{head}\PY{p}{(}df0\PY{p}{)}
\end{Verbatim}


    \begin{tabular}{r|llll}
 Endpoint Name & Last Checkin Time & Initial Install Date & Lifetime\\
\hline
	 ADHAYGAP21                      & Wed, 18 Jul 2018 11:36:29 -0700 & Wed, 27 Jul 2016 17:08:32 -0700 &  720 days                      \\
	 ADHAYVCR01                      & Wed, 18 Jul 2018 11:39:36 -0700 & Wed, 08 Jun 2016 10:33:56 -0700 &  770 days                      \\
	 ADHAYFDB01                      & Wed, 18 Jul 2018 11:03:06 -0700 & Sun, 20 Sep 2015 08:53:24 -0700 & 1032 days                      \\
	 ADHAYMCS01N1                    & Wed, 18 Jul 2018 11:31:05 -0700 & Sun, 20 Sep 2015 08:39:59 -0700 & 1032 days                      \\
	 ADHAYGAP16                      & Wed, 18 Jul 2018 11:40:01 -0700 & Thu, 16 Mar 2017 09:38:15 -0700 &  489 days                      \\
	 ADHAYWEB08                      & Wed, 18 Jul 2018 11:30:22 -0700 & Tue, 02 May 2017 07:43:19 -0700 &  442 days                      \\
\end{tabular}


    
    \begin{Verbatim}[commandchars=\\\{\}]
{\color{incolor}In [{\color{incolor}196}]:} str\PY{p}{(}df0\PY{p}{)}
\end{Verbatim}


    \begin{Verbatim}[commandchars=\\\{\}]
'data.frame':	141 obs. of  4 variables:
 \$ Endpoint Name       : chr  "ADHAYGAP21" "ADHAYVCR01" "ADHAYFDB01" "ADHAYMCS01N1" {\ldots}
 \$ Last Checkin Time   : chr  "Wed, 18 Jul 2018 11:36:29 -0700" "Wed, 18 Jul 2018 11:39:36 -0700" "Wed, 18 Jul 2018 11:03:06 -0700" "Wed, 18 Jul 2018 11:31:05 -0700" {\ldots}
 \$ Initial Install Date: chr  "Wed, 27 Jul 2016 17:08:32 -0700" "Wed, 08 Jun 2016 10:33:56 -0700" "Sun, 20 Sep 2015 08:53:24 -0700" "Sun, 20 Sep 2015 08:39:59 -0700" {\ldots}
 \$ Lifetime            :Class 'difftime'  atomic [1:141] 720 770 1032 1032 489 {\ldots}
  .. ..- attr(*, "units")= chr "days"
 - attr(*, "pandas.index")=RangeIndex(start=0, stop=141, step=1)

    \end{Verbatim}

    \begin{Verbatim}[commandchars=\\\{\}]
{\color{incolor}In [{\color{incolor}205}]:} df0\PY{o}{\PYZdl{}}l\PY{o}{\PYZlt{}\PYZhy{}}\PY{k+kc}{NULL}
\end{Verbatim}


    \begin{Verbatim}[commandchars=\\\{\}]
{\color{incolor}In [{\color{incolor}198}]:} str\PY{p}{(}df0\PY{p}{)}
\end{Verbatim}


    \begin{Verbatim}[commandchars=\\\{\}]
'data.frame':	141 obs. of  5 variables:
 \$ Endpoint Name       : chr  "ADHAYGAP21" "ADHAYVCR01" "ADHAYFDB01" "ADHAYMCS01N1" {\ldots}
 \$ Last Checkin Time   : chr  "Wed, 18 Jul 2018 11:36:29 -0700" "Wed, 18 Jul 2018 11:39:36 -0700" "Wed, 18 Jul 2018 11:03:06 -0700" "Wed, 18 Jul 2018 11:31:05 -0700" {\ldots}
 \$ Initial Install Date: chr  "Wed, 27 Jul 2016 17:08:32 -0700" "Wed, 08 Jun 2016 10:33:56 -0700" "Sun, 20 Sep 2015 08:53:24 -0700" "Sun, 20 Sep 2015 08:39:59 -0700" {\ldots}
 \$ Lifetime            :Class 'difftime'  atomic [1:141] 720 770 1032 1032 489 {\ldots}
  .. ..- attr(*, "units")= chr "days"
 \$ l                   : Factor w/ 78 levels "29","226","274",..: 41 51 73 73 30 19 61 53 39 45 {\ldots}
 - attr(*, "pandas.index")=RangeIndex(start=0, stop=141, step=1)

    \end{Verbatim}

    \begin{Verbatim}[commandchars=\\\{\}]
{\color{incolor}In [{\color{incolor}210}]:} \PY{k+kn}{library}\PY{p}{(}\PY{l+s}{\PYZsq{}}\PY{l+s}{ggplot2\PYZsq{}}\PY{p}{)}
          ggplot\PY{p}{(}df0\PY{p}{,} aes\PY{p}{(}x\PY{o}{=}\PY{k+kp}{as.factor}\PY{p}{(}Lifetime\PY{p}{)}\PY{p}{,} fill\PY{o}{=}\PY{k+kp}{as.factor}\PY{p}{(}Lifetime\PY{p}{)} \PY{p}{)}\PY{p}{)} \PY{o}{+}  geom\PYZus{}bar\PY{p}{(} \PY{p}{)} \PY{o}{+} coord\PYZus{}flip\PY{p}{(}\PY{p}{)} \PY{o}{+} labs\PY{p}{(}title\PY{o}{=}\PY{l+s}{\PYZdq{}}\PY{l+s}{Lifetime vs. Number of Endpoints (ADHAY...)\PYZdq{}}\PY{p}{,} x \PY{o}{=} \PY{l+s}{\PYZdq{}}\PY{l+s}{Lifetime (days)\PYZdq{}}\PY{p}{,} y\PY{o}{=}\PY{l+s}{\PYZdq{}}\PY{l+s}{\PYZsh{} of Endpoints\PYZdq{}}\PY{p}{)}
\end{Verbatim}


    
    
    \begin{center}
    \adjustimage{max size={0.9\linewidth}{0.9\paperheight}}{output_20_1.png}
    \end{center}
    { \hspace*{\fill} \\}
    

    % Add a bibliography block to the postdoc
    
    
    
    \end{document}
